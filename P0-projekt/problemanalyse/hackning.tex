\subsection{Sikkerhedskritisk software}
Ved at køre i de elektroniske køretøjer er der mange sikkerhedsmæssige fordele, i det at mennesker laver fejl som koster mange menneskers liv. Ved at gøre forskellige funktioner elektroniske mindsker man menneskelige fejl. Dog er der visse usikkerheder ved at lægge sin lid til maskinen alene. Usikkerheden ligger blandt andet i, at man som i alle andre elektroniske systemer kan `hacke' sig ind og ændre ved systemet. For eksempel er det muligt, med en relativ simpel teknik at kunne fjernstyre køretøjet. Det gør det muligt for `hackeren', at forstyrre kørselen og gøre det direkte farligt at køre. Dette er heldigvis ikke et udbredt problem endnu.  

Da producenterne gerne vil skabe en illusion om sikkerhed; hvad angår de elektroniske og automatiske køretøjer. Derfor bliver der lagt rigtig meget tid af til forskning i ikke at gøre denne usikkerhed til en virkelighed. Forskningen bliver foretaget af forskellige sikkerhedsforskere, som forsøger at påvise så mange fejl og usikkerheder ved køretøjet som muligt. Dette foregår ved at de selv `hacker' bilerne, og på den måde påviser under kontrollerede forhold, hvilke ricisier der er ved at automatisere flere af køretøjets funktioner.  


Et eksempel er sikkerhedsforskeren Charlie Miller og direktøren for IOactive Chris Valasek som har brugt over et års forskning på at undersøge og påvise hvordan man kan overtage kontrollen af en Jeep ved hjælp af en såkaldt zero-day exploit. Under forsøget opdagede de en fejl ved infotainment systemet, Harman uConnect, at internetforbindelsen igennem netværket Sprint, havde porten 6667 stående åben. Dette gav dem mulighed for at koble sig til bilen via deres smartphone over det cellulære netværk. Charlie og Chris var via en femtocell i stand til at fjernstyre Jeepen helt op til 110km. væk. Under angrebet på Jeepen var de i stand til blandt andet at styre rattet under bakkemanøvre, sætte bremserne ud af funktion og skifte gear, herudover nogle mindre ting som at styre klimaanlægget, sædevarmen, radioen, vinduesviskerne og sprinklervæsken.  

Chris og Charlie har efterfølgende indberettede fejlene de fandt under forskningen til Chrysler og Sprint, som var hurtige til at få rettet fejlene og få lukket den åbne port. Herudover fik Chrysler kaldt de fejlproducerede biler tilbage og fik lavet en opdatering som blev tilgængelig for ejerne af bilerne, hvor de selv kunne udføre opdateringen af bilen.   

Et andet eksempel er en nylig forskning der blev udført af Kevin Mahaffey som er medstifter af mobilsikkerhedsfirmaet Lookout, og Marc Rogers som er sikkerhedsforsker for CloudFlare. De har over to år forsket i den elektroniske arkitektur i en Tesla model S. Her fandt de to sårbarheder i systemet, som begge krævede at man til at starte med, havde fysisk adgang til køretøjet og adgang til køretøjets infotainment system, som er det, der styrer tænding og slukning af bilen. Udover det opdagede forskerne også at infotainment systemet benyttede sig af en gammel browseropdatering, som havde en fire år gammel sårbarhed, der gjorde det muligt for en potentiel hacker at udføre et angreb fuldstændigt uden fysisk adgang til bilen. Teoretisk set kunne en hacker lave en ondsindet hjemmeside, som gav ham adgang til infotainment systemet hvis en ejer af en Tesla besøgte hjemmesiden fra sin bil. Dog blev denne metode ikke testet af Kevin og Marc, men at finde sådan en sårbarhed i systemet er ikke utænkeligt da Tesla for nyligt har udgivet en opdatering til deres Tesla model S, som netop skulle forhindre en lignende type usikkerhed.  