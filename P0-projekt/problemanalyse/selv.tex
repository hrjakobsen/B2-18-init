
\subsection{Kommunikation mellem selvkørende biler}

I trafikken lige nu har hver trafikant sin egen ``kørestil'' som bestemmer hvor hurtig man kører, sin placering på vejen, hvornår man holder tilbage for andre, orientering, osv. Så at lave et system der kan styre alle disse biler som \'en enhed, vil gøre trafikken meget mere flydende og sikker. Hvis bilerne kunne kommunikere med hinanden, ville de kunne forudsige hvilke handlinger de andre biler tager. Derudover kan en forankørende bil sende informationer, om hvad der er forude til bilerne bagved. Informationen de giver hinanden vil være ting såsom, position, fart, hjulenes drejeposition, bremser og større billede af nærområdet. Disse ting hjælper den enkelte bil med at danne et billede af hvad der sker i forud i trafikken, som hjælper systemet i den enkelte bil til at køre mere effektivt i forhold til omgivelserne. 

Nu hvor vi kigger på kommunikationen mellem bilerne, er det oplagt at fejlagtige informationer kan blive sendt til andre biler, hvilket vil kunne `'forvirre' softwaren i disse biler. Vi skal huske at  softwaren kommer fra mennesker, og derfor kan der også opstå fejl I systemet. Alt elektronik kan slå fejl på et tidspunkt, så det vil ikke være en overraskelse, hvis der var en sjælden gang eller to, hvor bilen kører galt eller i hvert fald kommer ud for tekniske problemer der kan forårsage problemer. En anden ting som kunne være farlig er vejret, da det kan have en effekt på bilens sensorer. For at bilerne ikke bliver blændet af sådanne forhold, kunne det hjælpe at bilen havde mere kontakt med omverdenen. F.eks. hvis det regner og stormer en aften, så bliver bilens sensorer knap så gode til at se langt (ligesom mennesket). Her vil det være en fordel hvis den fik forskellige informationer om andre biler, hvilket vil hjælpe på sikkerheden. Men et problem der er værd at tænke på, er hvad hvis kommunikationen forsvinder? Altså den selvkørende bil skal først og fremmest selv kunne orientere sig i trafikken, så idet den mister sin kommunikationsevne skal den egentlig bare opføre sig normalt i trafikken. Den skal altså selv kunne orientere sig rundt i trafikken. Kommunikationsevne med andre biler er meget om at få information om blinde vinkler og uforudsete vejspærringer på en rute. Det med at kende til andre bilers position, fart osv. er en ekstra fordel for sikkerheden. En anden ting vi skal huske på er, at når man vælger at digitale systemer skal kunne kommunikere med hinanden, vil det altid være muligt at kunne hacke eller blande sig i informationen der bliver sendt mellem disse systemer. \cite{car_to_car}
