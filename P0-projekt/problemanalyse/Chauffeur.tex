\subsection{Googles Chauffeur}

National Highway Traffic Safety Administration har annonceret, et forslag til at alle fremtidige biler, skal komme med et system ved navn Vehicle To Vehicle eller V2V, som skal udsende ``beacons'' som indeholder information om bilens placering til andre biler i nærheden. V2V vil advare chaufføren, når der er fare for kollision, eksempelvis ved hård opbremsning. Bilproducenter kan så selv bestemme hvordan de vil behandle dataet fra V2V-systemet \cite{V2V}.

Googles AI, navngivet ``Chauffeur'', er på nuværende tidspunkt et system, som stadig kræver at man sidder bag rettet for at overtage styringen i tilfælde af miljøer, som ``Chauffeur'' ikke er trænet i. Ifølge et patent, vil der være en pind eller en knap, som tillader at slå ``Chauffeur'' til eller fra. Vælger man at slå ``Chauffeur'' til, vil softwaren først beslutte om ens GPS signal og det downloadede kort indeholder nok detaljer og bilen vil så enten begynde at køre eller meddele at tjenesten ikke er tilgængelig \cite{GooglePatent}.

Processen i at udvikle ``Chauffeur'' er at give bilen scenarier fra den virkelige verden, ved at holde en lukket beta, og køre rundt mens teknikere kan lave observationer omkring systemet, og kan tage over i værste tilfælde. Data viser at versioner af ``Chauffeur'' fra 2013 begår fejl 1 gang for hver 58.000 km \cite{PopSci}. Disse fejl betyder ikke at bilen vil køre galt, men nærmere at softwaren over- eller misfortolker objekter, og måske kategoriserer dem som noget bilen skal være ekstra eller mindre opmærksom på. At objekterne bliver misfortolket betyder ikke nødvendigvis at bilen kører galt, men det kan selvfølgelig lede til alvorlige problemer, da bilen ikke er sikker på hvad der findes rundt om bilen. Det farlige er dog at bilen ikke ved at der er noget galt, og vil køre efter disse forkerte observationer.

Indtil videre tager det gennemsnitligt 17 sekunder for en person at overtage fuld styring, hvis bilen eksempelvis ikke ved hvor den er. Bilen vil også have sensorer, som kan mærke at føreren tager fat i rattet, og det vil derfor ikke være nødvendigt at manuelt slå ``Chauffeur'' fra. ``Chauffeur'' er desuden også i stand til at detektere en sovende eller beruset fører, og kan i en sådan situation overtage kontrollen og forhindre et uheld \cite{GooglePatent}.

