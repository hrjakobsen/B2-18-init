\subsection{State of the Art} 
I 2015 under Consumer Electronics Show (CES), viste tre firmaer hver en bil frem, som viser hvor langt de hver i ser er, i udviklingen af selv-kørende biler.\cite{CES}
\subsubsection{Audi A7}
\begin{figure}[h!]
	\centering
	\includegraphics[width=0.8\textwidth]{images/150106_0345_ces.jpg}
	\captionsource{Audi A7 på 2015 CES}{\url{https://a248.e.akamai.net/f/574/7105/8d/www.extremetech.com/wp-content/uploads/2015/01/150106\_0345\_ces.jpg}}
	\label{fig:Audi_A7}
\end{figure}
Audi viste deres A7 frem, hvilken af de tre ligner mest en almindelig personbil, men den kan selv køre over lange distancer. Hvis den dog opdager mange mennesker i et område, typisk hvis den kører imod en by, vil den bede personen bag rattet om at overtage styrringen. Bilen havde inden konferencen kørt med nye trafikkanter og journalister fra Silicon Valley til Las Vegas, en distance på godt 900km, næsten uden input fra føreren og dette selvom bilen kørte 110km/t. Bilen benytter sig af sensorer som allerede bliver fabrikeret, som også er i stand til præcist at se bilens omgivelser. Sensorene inkluderer adaptiv fartkontrol, blindspots detektering og vejafvigring varsling, tre typer sensorer som allerede bliver brugt i diverse fartøjer i dag. Desuden kommer bilen også med andre sensorer så som laser skannere og kameraer, som gør bilen i stand til bedre at opdage objekter, både hvis objekterne er i bevægelse og hvis de er stillestående. Kameraerne, som optager i 3D, hjælper bilen med at holde styr på omkringværende traffik. Bilen er i stand til at køre på veje som ikke har høje bakker og vil ikke bringe dig ud i uheld, hvis ingen biler kører ind i din kørebane pludseligt, eller en bil foran panikbremser.
\subsubsection{Mercedes-Benz F 015}
Bilen her vil dog ikke komme i produktion da mulighed for at holde øje med vejen for passagerne er alt for få, men den indeholder teknologi og idéer ikke set før. F 015 har LED på hver side som hjælper fodgængere med at se, om det er sikkert for dem at gå ud foran bilen, en slags alternativ og en mulig løsning til fodgængerfelter og trafiklys i dag.
\begin{figure}[h!]
	\centering
	\includegraphics[width=0.8\textwidth]{images/150106_0422_ces.jpg}{}
	\captionsource{Mercedes-Benz konceptbilen F 015}{\url{https://a248.e.akamai.net/f/574/7105/8d/www.extremetech.com/wp-content/uploads/2015/01/150106\_0422\_ces.jpg}}
	\label{fig:Mercedes-Benz_F_015}
\end{figure}
Fodgængere kan potentielt krydse vejen hvor de har lyst, hvis bilerne på vejen indikerer det er sikkert nok for dem. Forsæderne i bilen vender imod bagsæderne, hvilket gør alle folk i bilen, i stand til at holde øjekontakt under samtaler. Sæderne har også mulighed for at rotere horisontalt, så passagerne nemt kan stige ind og ud af bilen.
\subsubsection{BMW i3}
\begin{figure}[h!]
	\centering
	\includegraphics[width=0.8\textwidth]{images/bmw-i3-fluid-black-1.jpg}
	\captionsource{Den selv-parkerende BMW i3}{\url{https://bmwactu.files.wordpress.com/2015/09/bmw-i3-fluid-black-1.jpg}}
	\label{fig:BMW_i3}
\end{figure}
BMW tog en eksisterende bil og modificerede den indenfor et helt bestemt område; parkering. Bilen her er i stand til at parkere sig selv, og ikke ligesom vi i dag kender det, hvor bilen parallelparkerer. Kør hen til en parkeringspladsindgang, gå ud af bilen, og BMW i3 vil selv køre rundt på parkeringspladsen indtil den finder en fri plads, parkere derefter låse og slukke sig selv. Skulle du få brug for bilen igen, kan du bruge trykke på skærmen inde i BMWs smartphone app til bilen, og så bare vente ved indgangen til parkeringspladsen, hvor bilen selv vil køre hen og samle dig op. Ikke kun skal bilen være i stand til at navigere rundt, den skal også være i stand til at undgå andre bilister, parkerede biler og ikke parkere på parkeringspladser prioriteret til handicappede.