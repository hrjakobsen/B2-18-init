\subsubsection{Teknologiens effekt}
Da de selvkørende biler som navnet antyder selv kører, bliver der i den forbindelse selvfølgelig skrevet en del kode til dem, for at sikre de kører ordentligt i trafikken. Dette betyder at bilen overholder de forskellige færdselslove, hvilket i sig selv er godt nok, men dette vil også have nogle seriøse økonomiske konsekvenser blandt andet for det offentlige, men også for de mange private virksomheder. 

Mange af de problemmatikker de selvkørende biler kan skabe, kan findes i den offentlige sektor. Det er som udgangspunkt lige til at finde ud af, hvordan de vil påvirke politiet og den indkomst staten har derfra. Da den selvkørende bil er programmeret til at overholde trafiklovene, betyder dette at der vil blive uddelt færre bøder. Politiet vil derfor komme til at have en lavere indtjening, da en stor del af denne nemlig kommer fra uddeling af bøder \cite{B}. Der vil heller ikke være brug for lige så mange betjente rundt omkring i landene, især ikke hvis alle biler bliver erstattet af selvkørende biler. Dette er fra statens side positivt, da det betyder at der er færre, som skal have løn, men det betyder derfor også, at der kommer flere arbejdsløse, hvilket er dårligt for økonomien.

Inden for det offentlige vil der selvfølgelig også være en mulighed for at skifte de offentlige transportmidler ud med selvkørende versioner. Dette vil på mange måder være en fordel for det offentlige, da det vil være billigere, fordi der ikke er en fører der skal have løn. Det vil derudover også være muligt for den at arbejde sammen med de andre selvkørende køretøjer, for eksempelvis at optimere den tid det vil tage at bruge offentligt transport. På den anden side vil der kunne opstå problemer, når computeren skal kende forskel på dem der skal med og dem der ikke skal. Dette kan tænkes som et problem, der kan løses med et stykke kode, som er i høj grad noget der skal overvejes før dette kan blive en realitet \cite{BUS}.

Udover staten vil mange private virksomheder, som for eksempel taxier, lastbiler og mange andre former for transport, også kunne mærke denne ændring til selvkørende biler. Man er allerede i New York begyndt at eksperimentere med selvkørende taxier, hvilket for firmaerne både kan spare tid og penge. De regner eksempelvis med at indføre 5.000 selvkørende taxier i 2016, hvilket også vil betyde at man kan forvente 5.000 arbejdsløse mere på gaderne i New York \cite{TAXI}. \\Derudover er der selvfølgelig også lastbilerne, som transporterer en masse forskellige varer rundt omkring over alt i verden. Dette kan nemlig også gøres ved hjælp af selvkørende biler, heraf lastbiler. Man bruger allerede selvkørende lastbiler, der hører inde i niveau 3 kategorien af SAE Internationals J3016 standard, hvilket gør det muligt at fjerne hænderne fra rettet på motorvejen. Derfor burde det ikke vare længe, før der slet ikke er brug for nogle bag rettet \cite{TRUCKS}. 

Et sidste område hvor ændringen til selvkørende biler vil skabe problemer, er for de mange tankstationer og andre sælgere at de fossile brændsler. De selvkørende biler vil sandsynligvis være elektriske, da fokusset på miljøvenlig teknologi er høj. Dette vil have en stor effekt på markedet for fossile brændsler, men også skabe en del arbejdsløse i forbindelse med arbejderne på de forskellige boreplatforme til de unge der står bag disken på den lokale tankstation \cite{GAS}.
