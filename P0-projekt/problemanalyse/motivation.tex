Selvkørende biler er meget oppe i medierne i disse dage, og man kan spørge sig selv hvorfor dette er tilfældet. Hvis man spørger producenterne af teknologien, får man svaret; at de selvkørende biler vil gøre trafikken mere sikker, samt vil udnytte vejene meget bedre \cite{GOOG_SITE}. I den forbindelse spurgte vi os selv: ``Jamen hvis de har så mange tilsyneladende fordele, hvorfor er bilerne så ikke godkendt til at køre rundt på vejene?''. 

Det er relevant at undersøge, om der er et ønske fra befolkningen om at få de selvkørende biler, da de også er med til at drive forskningen frem ved at skabe efterspørgsel efter teknologien. Efterspørgslen vil presse politikerne til at tage stilling til, om bilerne skal godkendes. Forskere fra Transportation Research Institute hos University of Michigan har undersøgt, hvor stor en del af befolkningen ønsker de selvkørende biler \cite{UMTRI}. Svaret var at 60\% ud af de 505 spurgte billister, har et ønske om enten delvist selvkørende biler eller helt selvkørende biler. De delvist selvkørende biler er typen som Google er ved at udvikle hvor føreren kan overtage kontrollen, hvorimod de fuldstændigt selvkørende biler ikke har brug for en fører.

Den store efterspørgsel efter de selvkørende biler, giver god grund til at undersøge dette initierende problem, og finde årsagerne til den manglende godkendelse af bilerne.