\subsubsection{Niveauer af automatisering af biler}
SAE International, tidligere Society of Automotive Engineers, er en organisation som fastsætter nye standarder indenfor automobil-industrien. Ifølge deres standard J3016 udgivet januar 2014, kan biler inddeles i 6 niveauer af automatisering, fra niveau 0 hvor der ingen automatisering er, til niveau 5 hvor alle funktioner i bilen er automatiseret \cite{SAE_J3016}. 

De fem niveauer kan beskrives som følger:

\begin{description}
	\item[Niveau 0] betyder ingen automatisering. Det vil sige at det er føreren af bilen der tager stilling til alle situationer, og selv har den fulde kontrol over bilen.
	\item[Niveau 1] betyder at bilen har kontrol over enten styring eller acceleration/deacceleration, mens føreren styrer de andre opgaver. Det er stadig føreren, der skal være opmærksom og fortælle bilen hvad den skal gøre. Dette niveau kan f.eks. være en fartpilot, hvor føreren blot bestemmer en fart, og så accelererer bilen op til denne hastighed, og holder derefter farten. 
	\item[Niveau 2] betyder at bilen kan styre bilen uden føreren, dog kun i nogle meget begrænsede tilfælde. Det er stadig føreren af bilen der skal holde øje med omgivelserne. Det vil sige at niveau 2 er delvis automatisering, altså automatisering af nogle få og specifikke opgaver i kørslen.
	\item[Niveau 3] er hvor man kan sige at en bil er selvkørende. Fra niveau 3 og fremefter, er det systemet der holder øje med omgivelserne, og analyserer forhindringer som bilen skal tage stilling til. På niveau 3 skal der dog være en fører af bilen der sidder klar til at overtage kontrollen af bilen, hvis bilen ikke kan genkende den situation den befinder sig i.
	\item[Niveau 4] og fremefter, har selv en fallback løsning hvis bilen oplever en ukendt situation. Dog gælder det stadig for niveau 4 at bilen ikke kan køre i alle situationer, og at brugeren stadig skal overtage i nogle bestemte situationer. Altså er bilen kun selvkørende i nogle scenarier, men i disse scenarier har den ikke brug for brugerens input.
	\item[Niveau 5] er fuld automatisering af bilen, og bilen har ikke længere brug for en fører af bilen. Den leverer selv fallback løsninger, og kan køre i alle kørselsscenarier. Dette er niveauet hvor man ikke har brug for et kørekort for at køre bilen, da den selv styrer alt, fra du sætter dig ind i bilen og til at du er fremme.
\end{description}

Den virksomhed som er længst med udviklingen af selvkørende biler, og får mest opmærksomhed i medierne er Google. Googles projekt for selvkørende biler, er lige nu en niveau 3 selvkørende bil, da der bag rattet skal sidde en billist med kørekort, klar til at overtage styringen hvis der går noget galt for programmet. Det vil sige at den AI som styrer bilen holder styr på omgivelserne og kører potentielt uden brug for billisten, men kan også advare billisten om at denne skal overtage styringen af bilen.

Googles selvkørende bil virker ved, at den på taget af bilen har en laser, som bruges til at scanne omgivelserne, og tegner et 3D kort over omgivelserne. Teknologien, der bruges til dette hedder Lidar, som er en sammentrækning af ordene ``light'' og ``radar'', virker ved at laseren peger på et objekt og analyserer det lys der kommer tilbage \cite{Lidar}. Jo længere tid det tager for lyset at komme tilbage, jo længere er objektet fra laseren. Ved at gøre dette på omgivelserne omkring bilen, kan der tegnes et detaljeret 3D kort over bilens position. Desuden bruger bilen også radarer samt kameraer til at få et overblik over hvor den befinder sig. Alle oplysninger fra disse sensorer behandles af en computer, som så ved hjælp af kunstig intelligens kan beslutte hvordan den skal reagere på dens situation. Den software som behandler dataet og træffer beslutninger hedder Google Chauffeur, hvilken vi kommer tilbage til senere \cite{GooglePatent}.

\subsubsection{Kunderne til de selvkørende biler}
Grunden til at det er relevant at kigge på disse selvkørende biler og deres udfordringer, er at der er mange potentielle kunder for denne teknologi. Når teknologien har mulighed for at blive udbredt mange steder, er det vigtigt at kende til de fordele og ulemper teknologien bringer, inden teknologien tages i brug.

Netop denne teknologi har potentiale til at blive udbredt på markedet, da en selvkørende bil eksempelvis vil være i stand til at spare mange menneskers tid hver morgen. Tiden i myldretrafikken ville da kunne benyttes på at arbejde, frem for at køre bil. Desuden lover producenterne af disse biler, også at disse biler vil øge trafiksikkerheden \cite{GOOG_SITE}, hvilket ville være endnu et incitament for en privat kunde at købe en sådan bil.

Udover den åbenlyse kunde i form af private mennesker er der også virksomheder, som ville kunne tænkes at bruge denne teknologi i deres forretningsmodel. Dette kunne f.eks. være et taxi-selskab der nu ikke nødvendigvis har brug for chauffører, da en computer nu kan styre bilen i stedet. Busser kunne også være et område, hvor man kunne benytte denne teknologi. Man kunne her igen spare lønnen til chaufføren, og virksomhederne ville kunne spare penge.

Som man kan se ud fra de ovennævnte punkter, er der mange muligheder for hvordan denne teknologi vil kunne blive brugt at både private mennesker og virksomheder. Når en teknologi bliver spredt i verden på denne måde, bliver vi nødt til at kigge nærmere på sikkerheden omkring den. Vi vil derfor i de følgende afsnit fokusere på om teknologien er klar til markedet endnu, samt kigge på om det er muligt at sikre sig, at disse biler er sikret over for hackere, der ønsker at skaffe kontrol over din bil. Et andet punkt som vi vil undersøge, er om hvilke indflydelse det vil have overfor vores hverdag, hvis disse biler blev en realitet.

\subsubsection{Producenterne af de selvkørende biler}
Udover kunderne, er der selvfølgelig også en anden gruppe mennesker der ønsker at få disse biler godkendt til vejene, nemlig producenterne af denne. Producenterne af bilerne bruger penge på lobbyister for at sikre sig at bilerne bliver godkendt til vejene \cite{soprweb}. Store bilfirmaer som Toyota, Mercedes og Audi har alle fremvist biler \cite{PopularMechanics} som kan beskrives som selvkørende biler. Disse firmaer har alle udtalt at de vil have biler på vejene med denne teknologi før 2020, og der er dermed sat et kapløb igang om hvem der først får sine biler på markedet. 