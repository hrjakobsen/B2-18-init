\subsection{Niveauer af automatisering af biler}
Når vi i de følgende afsnit vil behandle emnet selvkørende biler, er det nødvendigt først at tage stilling til hvornår en bil er selvkørende, og hvornår vi vil betegne køretøjet som værende selvkørende. SAE International er en organisation som fastsætter nye standarder inden for automobil-industrien. Ifølge deres standard J3016 udgivet januar 2014, kan biler inddeles i 6 niveauer af automatisering, fra niveau 0 hvor der ingen automatisering er, til niveau 5 hvor alle funktioner i bilen er automatiseret\cite{SAE_J3016}. 

Køretøjer på niveau 3 eller derover, bliver betegnet som værende selvkørende systemer. Det som udgør forskellen mellem niveau 2 og 3, er at på niveau 3 er det et computersystem som holder øje med omgivelserne, mens at på de tidligere niveauer har systemet kun styret funktioner inde i bilen, f.eks. i form af styring af retning eller hastighed.

Når vi når op på niveau 3 og systemet holder øje med omgivelserne, skal der dog stadig være en person klar til at overtage styringen hvis noget går galt. På niveau 4 er det systemet der leverer sin egen fallback, selv hvis brugeren ikke svarer på beskeder fra systemet om at overtage. Til sidst har vi så niveau 5, hvilket nok er det de fleste mennesker anser som en selvkørende bil. En bil hvor et system selv holder øje med omgivelserne, og selv besvarer på alle problemer og ændringer. Det vil sige at en sådan bil ikke er afhængig af at en person sidder i bilen. Når vi efterfølgende snakker om selvkørende biler i denne opgave, betyder det at vi snakker om niveau 3-5, da dette betyder at det er systemet der kan styre bilens enheder, samt at det er systemet der holder øje med omgivelserne.

\subsection{Kunderne til de selvkørende biler}
Grunden til at det er relevant at kigge på disse selvkørende biler og deres udfordringer er, at der er mange potentielle kunder for denne teknologi. Når teknologien har mulighed for at blive udbredt mange steder, er det vigtigt at kende til de fordele og ulemper teknologien bringer, inden teknologien tages i brug.

Netop denne teknologi har potentiale til at blive udbredt på markedet, da en selvkørende bil eksempelvis vil være i stand til at spare mange mennesker tid hver morgen. Tiden i myldretrafikken ville da kunne benyttes på at arbejde, frem for at køre bil. Desuden lover producenterne af disse biler også at disse biler vil øge trafiksikkerheden \cite{GOOG_SITE}, hvilket ville være endnu et incitament for en privat kunde at købe en sådan bil.

Udover den åbenlyse kunde i form af private mennesker, er der også virksomheder som ville kunne tænkes at bruge denne teknologi i deres forretningsmodel. Dette kunne f.eks. være et taxi-selskab der nu ikke nødvendigvis har brug for chauffører, da en computer nu kan styre bilen i stedet. Busser kunne også være et område hvor man kunne benytte denne teknologi. Man kunne her igen spare lønnen til chaufføren, og virksomhederne ville kunne spare penge.

Som man kan se ud fra de ovennævnte punkter, er der mange muligheder for hvordan denne teknologi vil kunne blive brugt at både private mennesker og virksomheder. Når en teknologi bliver spredt i verden på denne måde, bliver vi nødt til at kigge nærmere på sikkerheden omkring den. Vi vil derfor i de følgende afsnit fokusere på om teknologien er klar til markedet endnu, samt kigge på om det er muligt at sikre sig at disse biler er sikret over for hackere, der ønsker at skaffe kontrol over din bil. Et andet punkt som vi vil undersøge er, om hvilke indflydelse det vil have overfor vores hverdag, hvis disse biler blev en realitet. 