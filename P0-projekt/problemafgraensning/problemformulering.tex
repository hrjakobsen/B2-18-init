I afsnit \ref{afgraensning} har vi kigget nærmere på vigtigheden af at biler kan undersøge deres omgivelser, samt at de kan reagere forskelligt, alt efter hvilke typer af objekter er omkring dem. Bilen skal vide om det er en bil som holder parkeret i siden af vejen, eller om det er en bil der er ved at bakke ud af en indkørsel, ellers kan den overse vigtige informationer, som kunne lede til sammenstød. 

Vores initierende problem bestod i at undersøge hvorfor de selvkørende biler havde svært ved at blive godkendt til at køre på vejene. Igennem problemanalysen har vi undersøgt de problemer der forhindrer bilen i at blive godkendt til vejenene, og på baggrund af denne analyse kan vi nu opstille vores problemformulering.

Vores problemformulering lyder da:

\vspace{10 mm}
\noindent\makebox[\textwidth][c]{%
    \fbox{\begin{minipage}{0.7\textwidth}
	``Kan man lave et system der kan adskille forskellige trafikanter fra hinanden, og forudsige eventuelle pro-blemer som andre trafikanter kunne forårsage?''
\end{minipage}}}

\vspace{10 mm}

Vi kan desuden opstille følgende underspørgsmål der kunne arbejdes videre med:
\begin{itemize}
	\item Hvad er den mest effektive måde at genkende objekter på, for en computer?
	\item Hvordan kan en computer forudsige menneskers adfærd?
\end{itemize}