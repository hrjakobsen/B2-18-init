Vi har i afgrænsingen kigget nærmere på machine learning og mere specifikt på deep learning i form af neurale netværk. Man prøver i form af machine learning, at udvide selvkørende bilers forståelse af deres omgivelser og prøver at give dem en forståelse for trafikken. 

I den forbindelse kan man arbejde videre med at benytte dette machine learning på, at undersøge menneskers opførsel og opstille en model der kunne efterligne menneskers opførsel i trafikken. Ved hjælp af denne model vil en selvkørende bil, kunne forudsige menneskers opførsel i trafikken og tage højde for dette, når de selv træffer beslutninger om opførsel i trafikken. 

Dette kunne behjælpe de sammenstød, der indtil videre er sket med Googles selvkørende bil. Bilen ville nemlig vide, at personen bagved forventer, at Googles bil forsætter ud i det gule kryds.

Derfor kan vi opstille følgende spørgsmål, der kunne arbejdes med i en løsning:


\vspace{10 mm}
\noindent\makebox[\textwidth][c]{%
    \fbox{\begin{minipage}{0.7\textwidth}
Kan man ved hjælp af deep learning og neurale netværk, opstille en model der kan forudsige menneskers adfærd i trafikken? Kan man desuden ud fra data fra trafikken træne et neuralt netværk til at forudsige trafikanters adfærd?
\end{minipage}}}
