\subsection{Typer af machine learning}
Indenfor machine-learning findes forskellige metoder for at komme frem til data \cite{MachineLearning}. 
	\begin{enumerate}
		\item Supervised learning: 	Bliver ofte brugt til at foresige mulige begivenheder ved at bruge historisk data og kan derfor udregne hvornår folk muligvis vil bruge deres forsikring. Algoritmen får et set inputs og så sammenligner algoritmen dets egen output med det rigtige output for at finde fejl, og dermed forbedre sig selv.
		\item unsupervised learning: I modsætning til supervised bruges dette ikke indenfor historisk data og bliver heller ikke fortalt af en supervisor, hvilket svar er rigtigt. I stedet regnes der på dataen for at finde en struktur. Det kan bruges til markedsføring ved at sammenligne en masse data og finde lighed mellem disse data og kan derved være med til at påpege reklamer til et bestemt segment.
		\item semi-supervised learning: En blanding af de to tidligere. Algoritmen sammenligner godkendt og en helt del ikke-godkendt data, da det ikke-godkendte er billigere og nemmere at få adgang til. Bliver ofte brugt til at udpege den rigtige person, som blev fanget på et overvågningskamera, samt regression og forudsigelser.
		\item reinforcement-learning: Svarer lidt bruteforcing. Bliver ofte brugt indenfor robotter, navigation og gaming, og algoritmen fungerer gennem "trial and error", og vælger derefter den vej som giver bedst resultat.
	\end{enumerate}
Godt 70 \% af machine-learning foregår i dag gennem supervised learning og 20-30 \% foregår gennem unsupervised. Indenfor machine-learning findes andre metoder, som Data-mining og Deep learning, som begge har noget med machine-learning at gøre. Dagens og fremtidens selvkørende biler, benytter sig både af deep-learning teknikken samt semi-supervised learning\cite{Musk}. Både Tesla og Audi benytter sig allerede af Nvidia Drive PX, hvilket er et PCB, som tillader bilen det er installeret i, til at tilgå Nvidia's nye deep-learning platform kaldet DIGITS, hvilket alle enheder kan køre og tilgå, hvis de indeholder et Nvidia grafikkort. Dette gør det muligt for enheder, at lære deres omgivelser at kende, og er designet til at fungere ligesom et menneske. Ligesom mennesker lærer med tiden og af deres erfaringer, vil bilerne også lære af deres erfaringer, men da de alle er koblet til dette netværk, lærer alle bilerne, af alle bilers erfaringer\cite{Nvidia}. Dette board uploader sine data til, og hvis der er noget indsamlet data den ikke er sikker på hvad den skal gøre med, vil dette data blive kigget og regnet på af hele netværket.