\documentclass[12pt,hidelinks]{article}
\usepackage{graphicx,hyperref,amsmath,natbib,bm,url}
\usepackage{microtype,todonotes}
\usepackage[a4paper,text={14.5cm,23.2cm},centering]{geometry}
\usepackage[compact,small]{titlesec}
\usepackage[utf8]{inputenc}
\usepackage[nottoc,numbib]{tocbibind}

\clubpenalty = 10000
\widowpenalty = 10000
\usepackage[T1]{fontenc}
\hypersetup{
     colorlinks   = true,
     citecolor    = black,
     linkcolor    = black,
     urlcolor     = black
}
\renewcommand{\figurename}{Figur}
\renewcommand{\contentsname}{Indholdsfortegnelse}

\begin{document}
    \sloppy
    \titleGM
	\newpage
	\tableofcontents
	\newpage
	\section{Indledning}
	Selvkørende biler er meget oppe i medierne i disse dage, og man kan spørge sig selv hvorfor dette er tilfældet. Hvis man spørger producenterne af teknologien, får man svaret er, at de selvkørende biler vil gøre trafikken mere sikker, samt vil udnytte vejene meget bedre \cite{GOOG_SITE}. I den forbindelse spurgte vi os selv: ``Jamen hvis de har så mange tilsyneladende fordele, hvorfor er bilerne så ikke godkendt til at køre rundt på vejene?''. 

Man kunne spørge sig selv om der er et ønske fra befolkningen om at få de selvkørende biler, da de også er med til at drive forskningen frem ved at skabe efterspørgsel efter teknologien. Det vil presse politikkere til at tage stilling til om bilerne skal godkendes. Forskere fra Transportation Research Institute hos University of Michigan har undersøgt, hvor stor en del af befolkningen ønsker de selvkørende biler\cite{UMTRI}. Svaret var at 60\% ud af de 505 spurgte billister, har et ønske om enten delvist selvkørende biler eller helt selvkørende biler. De delvist selvkørende biler er typen som Google er ved at udvikle hvor føreren kan overtage kontrollen, hvorimod de fuldstændigt selvkørende biler ikke har brug for en fører.

Den store efterspørgsel efter de selvkørende biler, giver god grund til at undersøge dette initierende problem, og finde årsagerne til den manglende godkendelse af bilerne.
	\section{Problemanalyse}
	\textit{I dette afsnit vil vi undersøge hvilke problemer der står i vejen for at biler bliver godkendt til at køre på vejene. Vi har inddelt vores problemanalyse i to hovedpunkter, de teknologiske- og de samfundsmæssige udfordringer. På denne måde vil vi forsøge at få et overblik over problemet, og lokalisere de steder, hvor man kunne forbedre systemet så bilerne kan blive godkendt.}
	\subsection{Begrebsforklaringer}
	\subsubsection{Niveauer af automatisering af biler}
SAE International, tidligere Society of Automotive Engineers, er en organisation som fastsætter nye standarder indenfor automobil-industrien. Ifølge deres standard J3016 udgivet januar 2014, kan biler inddeles i 6 niveauer af automatisering, fra niveau 0 hvor der ingen automatisering er, til niveau 5 hvor alle funktioner i bilen er automatiseret \cite{SAE_J3016}. 

De fem niveauer kan beskrives som følger:

\begin{description}
	\item[Niveau 0] betyder ingen automatisering. Det vil sige at det er føreren af bilen der tager stilling til alle situationer, og selv har den fulde kontrol over bilen.
	\item[Niveau 1] betyder at bilen har kontrol over enten styring eller acceleration/deacceleration, mens føreren styrer de andre opgaver. Det er stadig føreren, der skal være opmærksom og fortælle bilen hvad den skal gøre. Dette niveau kan f.eks. være en fartpilot, hvor føreren blot bestemmer en fart, og så accelererer bilen op til denne hastighed, og holder derefter farten. 
	\item[Niveau 2] betyder at bilen kan styre bilen uden føreren, dog kun i nogle meget begrænsede tilfælde. Det er stadig føreren af bilen der skal holde øje med omgivelserne. Det vil sige at niveau 2 er delvis automatisering, altså automatisering af nogle få og specifikke opgaver i kørslen.
	\item[Niveau 3] er hvor man kan sige at en bil er selvkørende. Fra niveau 3 og fremefter, er det systemet der holder øje med omgivelserne, og analyserer forhindringer som bilen skal tage stilling til. På niveau 3 skal der dog være en fører af bilen der sidder klar til at overtage kontrollen af bilen, hvis bilen ikke kan genkende den situation den befinder sig i.
	\item[Niveau 4] og fremefter, har selv en fallback løsning hvis bilen oplever en ukendt situation. Dog gælder det stadig for niveau 4 at bilen ikke kan køre i alle situationer, og at brugeren stadig skal overtage i nogle bestemte situationer. Altså er bilen kun selvkørende i nogle scenarier, men i disse scenarier har den ikke brug for brugerens input.
	\item[Niveau 5] er fuld automatisering af bilen, og bilen har ikke længere brug for en fører af bilen. Den leverer selv fallback løsninger, og kan køre i alle kørselsscenarier. Dette er niveauet hvor man ikke har brug for et kørekort for at køre bilen, da den selv styrer alt, fra du sætter dig ind i bilen og til at du er fremme.
\end{description}

Den virksomhed som er længst med udviklingen af selvkørende biler, og får mest opmærksomhed i medierne er Google. Googles projekt for selvkørende biler, er lige nu en niveau 3 selvkørende bil, da der bag rattet skal sidde en billist med kørekort, klar til at overtage styringen hvis der går noget galt for programmet. Det vil sige at den AI som styrer bilen holder styr på omgivelserne og kører potentielt uden brug for billisten, men kan også advare billisten om at denne skal overtage styringen af bilen.

Googles selvkørende bil virker ved at den på taget af bilen har en laser som bruges til at scanne omgivelserne, og tegner et 3D kort over omgivelserne. Teknologien der bruges til dette hedder Lidar som er en sammentrækning af ordene ``light'' og ``radar'', virker ved at laseren peger på et objekt og analyserer det lys der kommer tilbage. Jo længere tid det tager for lyset at komme tilbage, jo længere er objektet fra laseren. Ved at gøre dette på omgivelserne omkring bilen, kan der tegnes et detaljeret 3D kort over bilens position. Desuden bruger bilen også radarer samt kameraer til at få et overblik over hvor den befinder sig. Alle oplysninger fra disse sensorer behandles af en computer som så ved hjælp af kunstig intelligens kan beslutte hvordan den skal reagere på dens situation. Den software som behandler dataet og træffer beslutninger hedder Google Chauffeur.

\subsubsection{Kunderne til de selvkørende biler}
Grunden til at det er relevant at kigge på disse selvkørende biler og deres udfordringer, er at der er mange potentielle kunder for denne teknologi. Når teknologien har mulighed for at blive udbredt mange steder, er det vigtigt at kende til de fordele og ulemper teknologien bringer, inden teknologien tages i brug.

Netop denne teknologi har potentiale til at blive udbredt på markedet, da en selvkørende bil eksempelvis vil være i stand til at spare mange mennesker tid hver morgen. Tiden i myldretrafikken ville da kunne benyttes på at arbejde, frem for at køre bil. Desuden lover producenterne af disse biler, også at disse biler vil øge trafiksikkerheden \cite{GOOG_SITE}, hvilket ville være endnu et incitament for en privat kunde at købe en sådan bil.

Udover den åbenlyse kunde i form af private mennesker er der også virksomheder, som ville kunne tænkes at bruge denne teknologi i deres forretningsmodel. Dette kunne f.eks. være et taxi-selskab der nu ikke nødvendigvis har brug for chauffører, da en computer nu kan styre bilen i stedet. Busser kunne også være et område hvor man kunne benytte denne teknologi. Man kunne her igen spare lønnen til chaufføren, og virksomhederne ville kunne spare penge.

Som man kan se ud fra de ovennævnte punkter, er der mange muligheder for hvordan denne teknologi vil kunne blive brugt at både private mennesker og virksomheder. Når en teknologi bliver spredt i verden på denne måde, bliver vi nødt til at kigge nærmere på sikkerheden omkring den. Vi vil derfor i de følgende afsnit fokusere på om teknologien er klar til markedet endnu, samt kigge på om det er muligt at sikre sig at disse biler er sikret over for hackere, der ønsker at skaffe kontrol over din bil. Et andet punkt som vi vil undersøge, er om hvilke indflydelse det vil have overfor vores hverdag, hvis disse biler blev en realitet.

\subsubsection{Producenterne af de selvkørende biler}
Udover kunderne, er der selvfølgelig også en anden gruppe mennesker der ønsker at få disse biler godkendt til vejene, nemlig producenterne af denne. Producenterne af bilerne bruger penge på lobbyister for at sikre sig at bilerne bliver godkendt til vejene\cite{soprweb}. Store bilfirmaer som Toyota, Mercedes og Audi har alle fremvist biler \cite{PopularMechanics} som kan beskrives som selvkørende biler. Disse firmaer har alle udtalt at de vil have biler på vejene med denne teknologi før 2020, og der er dermed sat et kapløb igang om hvem der først får sine biler på markedet. 
	\subsection{Teknologiske}
	I dette delafsnit vil vi kigge nærmere på de teknologiske problemer der er ved at få en selvkørende bil godkendt. Derfor kigger vi både på hvordan bilerne virker, og hvilke svagheder dette har, samt kigger på, hvordan hackning kunne vise sig at være en stor problemstilling inden for de selvkørende biler.
	\subsection{Bilernes interaktion med andre mennesker}
I takt med at der kommer flere  selvkørende biler på vejene, er der kommet mere lys på de problemer som disse biler har når de har interaktioner med omgivelserne. En af de større udfordringer de har mødt, er selve trafikken og alt der hører med til denne. Bilens sensorer gør at den har en større årvågenhed end andre bilister, hvilket faktisk har vist sig at være et problem. Problemet er, at bilens opmærksomhed også sætter den i stor fare for at blive påkørt af en uopmærksom bilist som kommer kørende bagfra. Dette bekræfter Google selv i deres månedlige rapport fra august 2015\cite{GOOG_MONTHLY}. 

En anden udfordring de selvkørende biler har er, at de som de er nu kører ekstremt sikkert. I dette tilfælde skal ordet sikkert ikke tages på en god måde. Den selvkørende bil er bilen der virker som om den er lige lidt for langsom ud af lyskrydset, og som altid sikrer sig at den ikke overstiger fartgrænsen på nogen måde. Dette gør at den ellers travle trafik bliver sinket af de selvkørende biler som er på gaden. Denne form for kørsel ville i princippet ikke være et problem for bilen, men da der ikke er nogen mennesker der kører på helt samme måde, så skaber det som sagt nogen problemer. Et eksempel kunne være som vist her:

\begin{figure}[h!]
    \centering
    \includegraphics[width=0.8\textwidth]{images/google_vision.png}
    \caption{Bilens scan, der viser hvad der er i nærheden. Altså dens syn}
    \label{fig:car_vision}
\end{figure}

På billedet ses en situation som er et eksempel på dens kørsel. Her skal de pink kasser ses som andre køretøjer, og den grønne linje som bilens rute, og midt i selve billedet er bilen selv. Men situationen her er, at bilen på den venstre side af den selvkørende bil svinger lidt bredt, hvilket gør at den kommer ind i vores selvkørende bils bane. Så i stedet for at selv tage et lidt bredere sving, så kan det ses at den selvkørende bil bremser ned. Det kan ses ved det røde hegn foran bilen, som betyder at bilen bremser. 

En anden situation hvor sådan en bil fejlede i at reagere blev beskrevet af en cyklist i Austin, Texas, hvor han nemlig mødte en sådanne selvkørende bil i et kryds, mens han var på cykel. Cyklisten lavede så det man kalder en track-stand, men han lagde sig ind bag bilen. En track-stand er en teknik man bruger til at holde sig på cyklen når man kører rigtigt langsomt, hvor man så også bevæger styret på cyklen frem og tilbage for at holde balancen. Den selvkørende bil misforstod så denne track-stand for en cyklist der kom kørende bag den og stoppede. Så da bilen kunne se at cyklisten holdte stille kørte den så igen, hvorefter den så spottede igen fordi cyklisten bevægede styret fra den ene side til den anden. Dette gjorde at bilen stadig ikke var nået ud til midten af krydset efter hele 2 minutters kørsel\cite{VOX}. Det lyder måske ikke af meget, men det er sådanne problemer der er nødvendig at fokusere på at sikre der ikke sker på de forkerte tidspunkter, da begge hændelser ville have kunne føre til større ulykker, eller yderligere forsinkelser af trafikken. 
	\subsubsection{Sikkerhedskritisk software}
Ved at køre i de automatiserede køretøjer er der mange sikkerhedsmæssige fordele, idet mennesker laver fejl som koster mange menneskers liv. Ved at gøre forskellige funktioner elektroniske mindsker man menneskelige fejl. Dog er der visse usikkerheder ved at lægge sin lid til maskinen alene. Usikkerheden ligger blandt andet i, at man som i alle andre elektroniske systemer kan `hacke' sig ind og ændre ved systemet.

Producenterne vil gerne skabe en sikkerhed, hvad angår de elektroniske og automatiske køretøjer. Der bliver derfor lagt meget forskning i at forhindre, at det bliver til en virkelighed. Forskningen bliver foretaget af forskellige sikkerhedsforskere, som forsøger at påvise så mange fejl og usikkerheder ved køretøjet som muligt. Dette foregår ved at de selv `hacker' bilerne, og på den måde påviser under kontrollerede forhold, hvilke risici der er ved at automatisere flere af køretøjets funktioner \cite{Jeep1}.

Et eksempel er sikkerhedsforskeren Charlie Miller og direktøren for IOactive Chris Valasek, som har brugt over et års forskning på at undersøge og påvise, hvordan man kan overtage kontrollen af en Jeep ved hjælp af en såkaldt zero-day exploit. Under forsøget opdagede de en fejl ved infotainment systemet, Harman uConnect, at internetforbindelsen igennem netværket Sprint, havde porten 6667 stående åben. Dette gav dem mulighed for at koble sig til bilen via deres smartphone over det cellulære netværk. Charlie og Chris var via en femtocell i stand til at fjernstyre Jeepen helt op til 110km væk. Under angrebet på Jeepen, var de i stand til blandt andet at styre rattet under en bakkemanøvre, sætte bremserne ud af funktion og skifte gear. Herudover nogle mindre ting som at styre klimaanlægget, sædevarmen, radioen, vinduesviskerne og sprinklervæsken \cite{Jeep1}\cite{Jeep2}.

Chris og Charlie har efterfølgende indberettet fejlene, de fandt under forsk-ningen til Chrysler og Sprint, som var hurtige til at få rettet fejlene og få lukket den åbne port. Herudover fik Chrysler kaldt de fejlproducerede biler tilbage og fik lavet en opdatering, som blev tilgængelig for ejerne af bilerne, hvor de selv kunne udføre opdateringen af bilen \cite{Jeep1}.  


Et andet eksempel er en nylig forskning der blev udført af Kevin Mahaffey, som er medstifter af mobilsikkerhedsfirmaet Lookout og Marc Rogers, som er sikkerhedsforsker for CloudFlare. De har over to år forsket i den elektroniske arkitektur i en Tesla Model S. Her fandt de to sårbarheder i systemet, som begge krævede at man til at starte med havde fysisk adgang til køretøjet og adgang til køretøjets infotainment system, som er et system som blandt andet kan styre om bilen er tændt eller slukket. Udover det opdagede forskerne også, at infotainment systemet benyttede sig af en gammel browser opdatering, som havde en fire år gammel sårbarhed, dette gjorde det muligt for en potentiel hacker at udføre et angreb helt uden fysisk adgang til bilen.

Teoretisk set kunne en hacker lave en ondsindet hjemmeside, som gav ham adgang til infotainment systemet hvis en ejer af en Tesla besøgte hjemmesiden fra sin bil.

Dog blev denne metode ikke testet af Kevin og Marc, men at finde sådan en sårbarhed i systemet er ikke utænkeligt, da Tesla for nyligt har udgivet en opdatering til deres Tesla Model S, som netop skulle forhindre en lignende type usikkerhed.  Formålet med den lange undersøgelse af Teslas køretøj var at se hvilke fejl Tesla og derfor også andre bilproducenter kunne have begået af sikkerhedsmæssige fejl \cite{Tesla}.


Et tredje eksempel, handler om de selvkørende bilers Lidar system og en nyligt sårbarhed opdaget af Jonathan Petit. Lidarsystemets funktion er at ``scanne'' alt, hvad der er rundt om bilen. Ved hjælp af en laserpointer og en Raspberry Pi er det muligt, at vildlede lidarsystemet til at opfange ikke eksisterende objekter. Denne ``hacking'' af lidarsystemet kan udføres i op til 100 meter radius fra bilen. Dette betyder, at man kan snyde bilen til fuld stop eller få den til at foretage en  undvigelsesmanøvre for fantomobjekter. Petit udtaler selv, at problemet ville kunne løses ved at have et system som, filtrerede alle umulige objekter fra ved hjælp af andre systemer så som radar \cite{Lidar}.

	
\subsection{Kommunikation mellem selvkørende biler}

I trafikken lige nu har hver trafikant sin egen "kørestil" som bestemmer hvor hurtig man kører, sin placering på vejen, hvornår man holder tilbage for andre, orientering, osv.  Så at lave et system der kan styre alle disse biler som \'en enhed, vil gøre trafikken meget mere flydende og sikker. Hvis bilerne kunne kommunikere med hinanden, ville de kunne forudse hvilken handling de andre biler tager. Derudover kan måske en forankørende bil sende informationer, om hvad der er forude til bilerne bagude. Informationen de giver hinanden vil være ting såsom, position, fart, hjulet drejeposition, bremser og større billede af nærområdet. Disse ting hjælper den enkelte bil at lave en billede af hvad der kan ske i trafikken forude, som hjælper systemet i den enkelte bil til at køre mere effektivt i forhold til omgivelserne. 

Vi skal dog også huske at det mennesket der har lavet de her biler, så de er ikke 100\% fejlfrie til evig tid. Alt elektronik kan slå fejl på et tidspunkt, så det vil ikke være en overraskelse, hvis der var en sjælden gang eller to, hvor bilen køre galt eller ihvertfald kommer ud for tekniske problemer der kan forsage problemer. En anden ting som kunne være farlig er vejret, da det kan have en effekt på bilens sensorer. For at bilerne ikke bliver blændet af sådanne forhold, kunne det hjælpe at bilen havde mere kontakt med omverdenen. Nu hvor vi kigger på kommunikationen mellem bilerne, er det oplagt at fejlagtige informationer kan blive sendt til andre biler, hvilket kunne `forvirre' softwaren i disse biler. En anden ting vi skal huske på er, at når man vælger at digitale systemer skal kunne kommunikere med hinanden, vil det altid være muligt at kunne hacke eller blande sig i informationen der bliver sendt mellem disse systemer. Disse falske stykker information der ville kunne blive sendt til bilerne ville svare til at en person fik bind for øjnene mens personen kørte bil. Det kunne have katastrofale følger. 

    \subsection{State of the Art}

	\subsection{Samfundsmæssige}
	I dette delafsnit vil vi fokusere på de samfundsmæssige problemer i forhold til de selvkørende biler. Disse problemer er problemer som politikkerne skal tage stilling til, samt problemer som kunne skræmme eventuelle købere væk.
	\subsubsection{Robotetik}
	Når der snakkes om selvkørende biler, bliver der ofte stillet etiske spørgsmål. Når man overfører kontrollen over bilen fra mennesket og overgiver dette til bilen selv, overføres også et stort ansvar. I tilfælde af at bilen kommer ud for et uheld, hvad skal bilen så gøre? Der er mange muligheder, men de kan ligeledes alle have store konsekvenser. Hvis en person ender foran bilen, skal den kunne træffe en beslutning om at forsøge at undvige personen og køre galt, eller om den blot skal forsøge at bremse. Hvis bilen bremser, er det ikke sikkert, at bilen kan nå at bremse for personen. Men hvis den forsøger at undvige, er det muligt, at den kører ind i med en modkørende bil. 

	Alt efter bilens programmering ville den kunne forsøge at minimere skader, ved at træffe det valg der vil forårsage færrest menneskeskader. Men dette kunne bringe uskyldige mennesker i fare, hvis bilen valgte at undvige en person ved at køre over i den anden side af vejen, hvor en modkørende bil kunne befinde sig. 
	
	Dette er et spørgsmål, som endnu ikke er besvaret fra de selvkørende biler, da dagens biler i en sådan situation vil overlade styringen til føreren. 

	En undersøgelse (se figur \ref{fig:etik_accident}) blev i 2014 foretaget af Open Roboethics Initiative, hvor mennesker blev adspurgt, om hvordan de ønskede at bilen skulle opføre sig i en situation som nævnt ovenfor. 52\% af de adspurgte mennesker ønsker, at bilen skal forsøge at minimere skaderne ved at fordele disse mellem både passagerer og fodgængere. 
	

	\begin{figure}[h!]
		\centering
		\includegraphics[width=\textwidth]{images/roboethics-2.jpg}
		\captionsource{Undersøgelse om hvad folk vil foretrække i en uundgåelig ulykke.}{\url{http://www.openroboethics.org/results-random-chance-over-informed-decision/}}
		\label{fig:etik_accident}
	\end{figure}

	I et scenarium, hvor et tog kommer kørende ned ad en bane imod tre mennesker spændt fast til togskinnerne, men inden toget støder mod disse tre personer, er der et baneskift. Hvis der på den anden række togskinner kun ligger \'en person, vil den logiske løsning være at skifte bane for toget, da der dermed reddes 2 menneskeliv. Men hvad hvis denne ene person er meget betydningsfuld for en, så som en mor eller ens egen søn, ville samme person så træffe samme valg og vælge at slå sit eget barn ihjel, frem for tre fremmede mennesker? En maskine vil stadig tænke logisk og træffe samme valg da færre liv går tabt, men ingen forældre vil have det komfortabelt med at benytte en bil, som kan træffe selv samme beslutning og slå deres eget barn ihjel, som muligvis sidder på bagsædet, frem for at ramme nogle fremmede fodgængere.
	
	Tilbage i 1942, præsenterede science fiction forfatteren Isaac Asimov, 3 gyldne love \cite{Asimov}, som siden er blevet brugt utallige gange til at beskrive, hvordan en robot hovedsageligt skal omgås mennesker.
	
	\begin{enumerate}
		
		\item En robot må ikke gøre et menneske fortræd, eller, ved ikke at gøre noget, lade et menneske komme til skade
		\item En robot skal adlyde ordrer givet af mennesker, så længe disse ikke er i konflikt med første lov
		\item En robot skal beskytte sin egen eksistens, så længe dette ikke er i konflikt med første eller anden lov
		
	\end{enumerate}
	
	Disse tre love er i bund og grund også etiske, da de siger at robotten ikke må adlyde et menneskes ordre, hvis dette forårsager skader på andre mennesker, men samtidig skal sørge for at mennesker ikke kommer til skade. Hvis et deprimeret individ springer ud foran en selvkørende bil i et forsøg på at begå selvmord, må bilen ikke bare køre personen over, da dette er i strid mod første lov. Bilen bliver derfor nødt til at undvige, også selvom dette kan totalskade bilen, og brække et legeme eller to på passageren, da et brækket ben er langt bedre end et mistet liv, samtidig med at overholde Asimov's 3 love. Her bliver føreren af bilen dog ``straffet'' for, at et andet menneske forsøger at tage sit eget liv. Skal bilen i dette tilfælde gøre det samme som føreren af en manuelt-kørende bil og køre ham ned da det er umuligt at undvige, selvom dette vil være i strid mod første lov? 

	Det kan være svært, at svare på disse spørgsmål og i sidste ende vil sådan nogle problemstillinger besvares, ved at politikere fastsætter love der giver producenterne nogle klare linjer at programmere efter.
	\subsubsection{Teknologiens effekt}
Da de selvkørende biler som navnet antyder selv kører, bliver der i den forbindelse selvfølgelig skrevet en del kode til dem, for at sikre de kører ordentligt i trafikken. Dette betyder at bilen overholder de forskellige færdselslove, hvilket i sig selv er godt nok, men dette vil også have nogle seriøse økonomiske konsekvenser blandt andet for det offentlige, men også for de mange private virksomheder. 

Mange af de problemmatikker de selvkørende biler kan skabe, kan findes i den offentlige sektor. Det er som udgangspunkt lige til at finde ud af, hvordan de vil påvirke politiet og den indkomst staten har derfra. Da den selvkørende bil er programmeret til at overholde trafiklovene, betyder dette at der vil blive uddelt færre bøder. Politiet vil derfor komme til at have en lavere indtjening, da en stor del af denne nemlig kommer fra uddeling af bøder\cite{B}. Der vil heller ikke være brug for lige så mange betjente rundt omkring i landene, især ikke hvis alle biler bliver erstattet af selvkørende biler. Dette er fra statens side positivt, da det betyder at der er færre, som skal have løn, men det betyder derfor også, at der kommer flere arbejdsløse, hvilket er dårligt for økonomien.

Inden for det offentlige vil der selvfølgelig også være en mulighed for at skifte de offentlige transportmidler ud med selvkørende versioner. Dette vil på mange måder være en fordel for det offentlige, da det vil være billigere, fordi der ikke er en fører der skal have løn. Det vil derudover også være muligt for den at arbejde sammen med de andre selvkørende køretøjer, for eksempelvis at optimere den tid det vil tage at bruge offentligt transport. På den anden side vil der kunne opstå problemer, når computeren skal kende forskel på dem der skal med og dem der ikke skal. Dette kan tænkes som et problem, der kan løses med et stykke kode, som er i høj grad noget der skal overvejes før dette kan blive en realitet\cite{BUS}.

Udover staten vil mange private virksomheder, som for eksempel taxier, lastbiler og mange andre former for transport, også kunne mærke denne ændring til selvkørende biler. Man er allerede i New York begyndt at eksperimentere med selvkørende taxier, hvilket for firmaerne både kan spare tid og penge. De regner eksempelvis med at indføre 5.000 selvkørende taxier i 2016, hvilket også vil betyde at man kan forvente 5.000 arbejdsløse mere på gaderne i New York\cite{TAXI}. Derudover er der selvfølgelig også lastbilerne, som transporterer en masse forskellige varer rundt omkring over alt i verden. Dette kan nemlig også gøres ved hjælp af selvkørende biler, heraf lastbiler. Man bruger allerede selvkørende lastbiler, der hører inde i niveau 3 kategorien af SAE Internationals J3016 standard, hvilket gør det muligt at fjerne hænderne fra rettet på motorvejen. Derfor burde det ikke vare længe, før der slet ikke er brug for nogle bag rettet\cite{TRUCKS}. 

Et sidste område hvor ændringen til selvkørende biler vil skabe problemer, er for de mange tankstationer og andre sælgere at de fossile brændsler. De selvkørende biler vil sandsynligvis være elektriske, da fokusset på miljøvenlig teknologi er høj. Dette vil have en stor effekt på markedet for fossile brændsler, men også skabe en del arbejdsløse i forbindelse med arbejderne på de forskellige boreplatforme til de unge der står bag disken på den lokale tankstation\cite{GAS}.

	\section{Problemafgrænsning}
	\textit{I dette afsnit vil vi indsnævre vores problemrum fra problemanalysen, og finde frem til fokusområdet for vores problemformulering, og for en eventuel løsning af et af problemerne vi har fundet i problemanalysen. Ud fra denne problemafgrænsning, vil vi så opstille en problemformulering.}.
    \label{afgraensning}
\subsection{Valg af fokusområde}
I vores problemanalyse har vi kigget på de teknologiske og samfundsmæssige problemer omkring teknologien. Der har været problemer af begge typer, men de problemer som har vist sig i form af ulykker for bilen, har været fordi bilen ikke har kørt som andre mennesker. Som nævnt i afsnit \ref{interaktion}, kører den selvkørende bil blandt andet ikke aggressivt nok, hvilket har forårsaget at biler er kørt op bag i den.

\subsection{Bilens syn}
Så hvordan undersøger bilen hvad der er rundt om den? Både Tesla og Audi benytter sig af Nvidia Drive PX, hvilket er en udviklingsplatform, som tillader bilen det er installeret i, at tilgå Nvidias nye deep-learning platform kaldet DIGITS. Dette gør det muligt for enheder, at lære deres omgivelser at kende, og er designet til at fungere ligesom et menneske. 

\begin{figure}[h!]
	\centering
	\includegraphics[width=\textwidth]{images/nvidiadrive.png}
	\captionsource{Nvidia DRIVE PX analyserer omgivelserne, og kan inddele objekter i forskellige grupper.}{\url{https://www.nvidia.com/object/drive-px.html}}
	\label{fig:DRIVE}
\end{figure}
Ligesom mennesker lærer med tiden og af deres erfaringer, vil bilerne også lære af deres erfaringer, men da de alle er koblet til dette netværk, lærer alle bilerne, af alle bilers erfaringer\cite{Nvidia}. Dette board uploader sine data til DIGITS, og hvis der er noget indsamlet data den ikke er sikker på hvad den skal gøre med, vil dette data blive kigget og regnet på af hele netværket.

DIGITS analyserer billeder fra kamerarer rundt om bilen, og inddeler objekterne rundt om bilen i klasser. På figur \ref{fig:DRIVE} kan man se hvordan objekter er klassificeret på et stilbillede.

DIGITS er selvfølgelig kun \'en platform, og andre producenter af selvkørende biler vil måske benytte sig af andre systemer. Men det vigtigste er blot at kunne skille de forskellige objekter fra hinanden\cite{cnet}, så ikke alt ses som den samme type objekter. Grunden til dette er, at bilen skal kunne forholde sig anderledes over for en cyklist, end for en anden bil. Programmet bliver nødt til at kende forskellen på en parkeret bil der holder i vejkanten, og en bil som kommer fra en sidevej og kan finde på at køre ud foran den selvkørende bil. Normaltvist vil et menneske forsøge at få øjenkontakt med en billist som kommer fra en sidevej, for at sikre sig at bilen er blevet set, men da en selvkørende bil ikke kan lave øjenkontakt, bliver den nødt til at kunne adskille de forskellige typer af objekter på vejene.

\subsection{Typer af trafikanter}
Det er vigtigt at en selvkørende bil kan adskille forskellige typer af trafikanter, da de hver især har forskellige karakteristika. I en rapport fra Havarikommissionen for vejtrafikulykker viser resultaterne, at størstedelen af de ulykker der skete mellem cyklister og billister skyldtes dårlige trafikvaner\cite{HVU}. I 17 ud af 30 undersøgt ulykker havde cyklisten vigepligt, men kørte alligevel ud i krydset. Undersøgelsen konkluderer at både cyklister og billister var opmærksomme, og at ulykkerne overvejende skyldtes dårlige trafikvaner. 

Hvad betyder det så for de selvkørende biler? Det betyder at hvis bilen opdager en cyklist, bliver den nødt til at vide at nogle cyklister har dårlige trafikvaner, og at den derfor skal være opmærksom på at cyklisten ikke nødvendigvis vil køre efter de gældende trafikregler. 

I forhold til billister har vi allerede i afsnit \ref{interaktion}	
    \subsubsection{Support Vector Machines}
\textbf{S}upport \textbf{V}ector \textbf{M}achine eller SVM er en gren inden for supervised machine learning\cite{ML_BOOK}, som går ud på at inddele datapunkter i forskellige klasser. I forbindelse med selvkørende biler, kunne SVM blive brugt til at separere billister fra cyklister, og igen fra fodgængere.

SVM virker ved at indsætte datapunkter i et koordinatsystem, og så finde et `hyperplane' der kan skille de forskellige typer data fra hinanden. På figur \ref{fig:SVM} ses et koordinatsystem hvor der er indtegnet to forskellige typer data, som ønskes adskilt. For at adskille de to typer data på figur \ref{fig:SVM}, kan der bruges en lineær funktion. Det kan ses på illustrationen, at linjen $h_1$ ikke adskiller de to datasæt, og derfor ville dette hyperplane ikke virke. $h_2$ adskiller de to datasæt, men med en meget lille margin. Til sidst har vi så $h_3$ som adskiller de to datasæt med en stor margin, her vil der altså være størst sandsynlighed for at en computer kunne forudsige hvilken gruppe datapunktet hører til. 
\begin{figure}[h!]
	\centering
	\includegraphics[width=0.6\textwidth]{images/SVM.png}
	\captionsource{Illustration af SVM}{\url{https://en.wikipedia.org/wiki/Support\_vector\_machine\#/media/File:Kernel\_Machine.png}}
	\label{fig:SVM}
\end{figure}

Kort sagt kan man altså sige at SVM går ud på at opstille et hyperplane med størst mulig afstand til de nærmeste datapunkter, da dette giver den største sandsynlighed for at computeren kan adskille dataet korrekt.
Når funktionen for hyperplanet er opstillet, vil den til en given vektor (som er datapunktet) give $h(\vec{x}) \geq 1$ hvis datapunktet ligger i den ene gruppe, hvorimod $h(\vec{x}) \leq -1$ hvis datapunktet ligger i den anden gruppe. 
    \subsection{Typer af machine learning}
Indenfor machine-learning findes forskellige metoder for at komme frem til data \cite{MachineLearning}. 
	\begin{enumerate}
		\item Supervised learning: 	Bliver ofte brugt til at foresige mulige begivenheder ved at bruge historisk data og kan derfor udregne hvornår folk muligvis vil bruge deres forsikring. Algoritmen får et set inputs og så sammenligner algoritmen dets egen output med det rigtige output for at finde fejl, og dermed forbedre sig selv.
		\item unsupervised learning: I modsætning til supervised bruges dette ikke indenfor historisk data og bliver heller ikke fortalt af en supervisor, hvilket svar er rigtigt. I stedet regnes der på dataen for at finde en struktur. Det kan bruges til markedsføring ved at sammenligne en masse data og finde lighed mellem disse data og kan derved være med til at påpege reklamer til et bestemt segment.
		\item semi-supervised learning: En blanding af de to tidligere. Algoritmen sammenligner godkendt og en helt del ikke-godkendt data, da det ikke-godkendte er billigere og nemmere at få adgang til. Bliver ofte brugt til at udpege den rigtige person, som blev fanget på et overvågningskamera, samt regression og forudsigelser.
		\item reinforcement-learning: Svarer lidt bruteforcing. Bliver ofte brugt indenfor robotter, navigation og gaming, og algoritmen fungerer gennem "trial and error", og vælger derefter den vej som giver bedst resultat.
	\end{enumerate}
Godt 70 \% af machine-learning foregår i dag gennem supervised learning og 20-30 \% foregår gennem unsupervised. Indenfor machine-learning findes andre metoder, som Data-mining og Deep learning, som begge har noget med machine-learning at gøre. Dagens og fremtidens selvkørende biler, benytter sig både af deep-learning teknikken samt semi-supervised learning\cite{Musk}. Både Tesla og Audi benytter sig allerede af Nvidia Drive PX, hvilket er et PCB, som tillader bilen det er installeret i, til at tilgå Nvidia's nye deep-learning platform kaldet DIGITS, hvilket alle enheder kan køre og tilgå, hvis de indeholder et Nvidia grafikkort. Dette gør det muligt for enheder, at lære deres omgivelser at kende, og er designet til at fungere ligesom et menneske. Ligesom mennesker lærer med tiden og af deres erfaringer, vil bilerne også lære af deres erfaringer, men da de alle er koblet til dette netværk, lærer alle bilerne, af alle bilers erfaringer\cite{Nvidia}. Dette board uploader sine data til, og hvis der er noget indsamlet data den ikke er sikker på hvad den skal gøre med, vil dette data blive kigget og regnet på af hele netværket.
    \section{Problemformulering}
	\bibliographystyle{plain}
	\bibliography{sources}
\end{document}