\documentclass[12pt,hidelinks]{article}
\usepackage[nottoc,numbib]{tocbibind}
\begin{document}
	\section{machine-learning analytics}
	http://www.sas.com/en\_us/insights/analytics/machine-learning.html
	Maskiner lærer allerede i dag ved at bruge algoritmer, som bliver brugt til at opdage snyd og bedrag, samt til hvad bliver smidt i ens email spamfilter, hvilke resultater skal vises når der bliver søgt på en søgemaskine, og rekognosceringteknikker indenfor mange områder.
	
	Indenfor machine-learning findes forskellige teknikker.
	Godt 70 procent machine-learning i dag foregår gennem supervised-, hvor 10-20 procent foregår gennem unsupervised-learning.
	\begin{enumerate}
		\item Supervised learning: 	Bliver ofte brugt til at foresige mulige begivenheder ved at bruge historisk data og kan derfor udregne hvornår folk muligvis vil bruge deres forsikring. Algoritmen får et set inputs og så sammenligner algoritmen dets egen output med det rigtige output for at finde fejl, og dermed forbedre sig selv.
		\item unsupervised learning: I modsætning til supervised bruges dette ikke indenfor historisk data og bliver heller ikke fortalt af en supervisor, hvilket svar er rigtigt. I stedet regnes der på dataen for at finde en struktur. Det kan bruges til markedsføring ved at sammenligne en masse data og finde lighed mellem disse data og kan derved være med til at påpege reklamer til et bestemt segment.
		\item semi-supervised learning: En blanding af de to tidligere. Algoritmen sammenligner godkendt og en helt del ikke-godkendt data, da det ikke-godkendte er billigere og nemmere at få adgang til. Bliver ofte brugt til at udpege den rigtige person, som blev fanget på et overvågningskamera, samt regression og forudsigelser.
		\item reinforcement-learning: Svarer lidt bruteforcing. Bliver ofte brugt indenfor robotter, navigation og gaming, og algoritmen fungerer gennem "trial and error", og vælger derefter den vej som giver bedst resultat.
	\end{enumerate}
	Datamining og machine-learning er ikke det samme. Datamining opdager nye og tidligere ukendt metoder og viden, hvor machine-learning bruger kendte metoder og viden til brug i anden data for at udregne nye resultater som i sidste ende giver bedre resultater, end hvad man havde tidligere. Datamining bliver brugt til at forbedre machine-learning
\end{document}
