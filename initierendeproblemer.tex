\documentclass{article}
\usepackage{graphicx,hyperref,amsmath,natbib,bm,url}
\usepackage{microtype,todonotes}
\usepackage[a4paper,text={14.5cm,23.2cm},centering]{geometry}
\usepackage[compact,small]{titlesec}
\usepackage[utf8]{inputenc}
\usepackage[nottoc,numbib]{tocbibind}

\clubpenalty = 10000
\widowpenalty = 10000
\usepackage[T1]{fontenc}
\hypersetup{
     colorlinks   = true,
     citecolor    = black,
     linkcolor    = black,
     urlcolor     = black
}
\renewcommand{\figurename}{Figur}
\renewcommand{\contentsname}{Indholdsfortegnelse}

\begin{document}
	\title{PV2}
	\author{Gruppe B2-18}
	\maketitle
	\section*{Initierende problemer}
	\begin{enumerate}
		\item Hvorfor har selvkørende biler svært ved at blive godkendt, når de har mange åbenlyse fordele? \label{problem}
		\item Er selvkørende biler sikre at have på vejene, når de skal have mulighed for at kommunikere med andre? (hacking etc.)
		\item Hvornår må software slå ihjel? (hvis bilen kommer ud for et tilfælde hvor den skal træffe valg mellem hvem den skal køre ned)
	\end{enumerate}

	\subsection*{Valg af initierende problem}
	Vi har valgt at arbejde med problem \ref{problem}, da selvkørende biler i sig selv er et forholdsvist snævert område. Derfor giver problem \ref{problem} os mulighed for at undersøge forskellige aspekter af de selvkørende biler. De to vil i mindre grad komme til udtryk i vores problemanalyse, da de begge spiller ind på de udfordringer de selvkørende biler står over for, når disse skal igennem en godkendelsesprocess.

	\section*{Analyse af problemfeltet}
	\begin{description}
		\item[Hvorfor? -] Det er et forholdsvist nyt felt, og man har ikke megen erfaring med disse typer biler på vejene. Få firmaer forsøger at skubbe udviklingen fremad, men mange love kræver også at politiske beslutninger skal tages, før end bilerne kan få lov til at køre på vejene.

		\item[Hvad? -] Mange trafikuheld forårsages af menneskelige fejl. Enten fordi folk er påvirket af stoffer eller alkohol, eller blot på grund af træthed og/eller fordi de ikke er fokuseret.

		I 2013 kom 2984 mennesker til skade i bilulykker \footnote{\url{http://www.statistikbanken.dk}}. Dette tal kunne nedsættes, hvis det var muligt at fjerne de menneskelige svagheder, eksempelvis i form af koncentrationsbesvær eller træthed.

		\item[Hvornår? -] Det er et aktuelt problem, da den nye teknologi stadig har en forholdsvis lang vej foran sig, før end systemet vil være klar til at blive brugt overalt i byerne. At bilerne ikke bliver godkendt til brug i byerne, forsinker denne test-periode og forlænger tiden inden de selvkørende biler er klar til at blive brugt.

		\item[Hvor? -] Geografisk set er problemet at finde i USA, da det er her de store virksomheder der arbejder med dette problem er lokaliseret. Mere specifikt ligger problemet dels hos virksomhederne der udvikler disse biler, og som har ansvaret for at få dem godkendt. Men problemet ligger også i de politiske cirkler, da det er politikerne der skal være åbne for den udvikling der forekommer og være med til at fremme teknologien.

		\item[Hvem? -] Problemet er forårsaget af producenterne, da teknologien ikke er færdigudviklet\footnote{\url{http://nyhederne.tv2.dk/tech/2015-09-02-googles-selvkoerende-bil-har-problemer-med-mennesker}}. Producenten vil selvfølgelig være sikre på at produktet er i orden før de vil sende dem ud, da et enkelt uheld ville kunne betyde enden for et projekt som dette, hvor en fejl kan betyde menneskeliv. Problemet er dog også i høj grad forårsaget af politikerne, da producenterne ikke kan teste alle scenarier på en bane, og i stedet bliver nødt til at komme ud i trafikken og teste bilerne under realistiske forhold. 

		De som rammes af problemerne er i sidste ende forbrugerne. Ved at hastigheden af udviklingen af teknologien sænkes, sænkes antallet af trafikuheld ikke. 

		Mindre firmaer der vil forsøge sig med teknologien, kan have svært ved at komme ind på markedet, da det kræver mange penge og projekterne kan blive standset af politiske beslutninger, ved at de ikke vil lade virksomhederne teste bilerne.

		\item[Hvordan? -] Lige nu forsøger virksomhederne at løse problemet ved at lave mere forskning, og gør bilerne mere og mere sikre, og sikrer sig at bilerne opfylder alle lovkrav. De få steder de kan teste bilerne, bliver der hele tiden kørt rundt og opsamlet data, hvilket kan bruges til at gøre bilerne mere sikre, og mere tilbøjelige til at blive godkendt flere steder rundt om i verden.
	\end{description}
\end{document}
