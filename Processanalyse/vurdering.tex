I gruppen skabte det forvirring at vi manglede en mand, og det fyldte meget i vores gruppediskussioner. Vi gjorde ikke så meget ved situationen, udover at kommentere på den.

På nogle tidspunkter igennem projektet manglede vi lidt struktur, og det var her ikke tydeligt hvad vi skulle gå i gang med. Dette forårsagede at vi på disse tidspunkter ikke var så fokuserede som vi kunne have været.

I gruppen var vi gode til at overholde de aftaler vi havde opstillet. Kontrakten blev kun brudt \'en gang, og personen som brød aftalen gjorde det kun den ene gang, men overholdte også den konsekvens som kontrakten foreskrev. 

Vores vejleder var rigtig god til at komme med vurderinger af det arbejde vi havde lavet. Dette gav os en forsikring om at vi kunne arbejde videre på vores emner, uden at det ramte uden for projektets grænser. Den ene misforståelse der skete, blev taget op på næste vejledermøde, og vi kunne hurtigt få rettet afsnittet om til at passe bedre på rapporten.

Det skabte som sagt forvirring i starten af projektet at skulle skrive det i \LaTeX og versionsstyringsprogrammet git, men efterhånden som projektet skred frem, lærte flere i gruppen at bruge disse, og hele gruppen var god til at hjælpe hinanden med problemer når disse opstod.

Da vi hurtigt kom i gang med projektet, var vi også forholdsvist hurtigt færdige. Selvom vi havde skrevet et langt beskrivende afsnit, som vi senere fandt ud af ikke passede på rapporten, var det ikke nødvendigt at stresse over projektet, og vi kunne nå at læse rapporten igennem flere gange inden aflevering. 