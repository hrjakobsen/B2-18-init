\subsection*{Projektplanlægningen}
For at vælge vores tema til projektet, skrev vi vores forslag op på tavlen. Derefter diskuterede vi i gruppen hvilke vi helst ville arbejde med, samt hvilke emner hvor man kunne se problemer at beskæftige sig med.

Grundet projektets størrelse, valgte vi ikke at lave en tidsplan. Derimod lavede vi mundtlige aftaler fra gang til gang om hvad der skulle laves. I starten af hver gruppesession snakkede vi om hvad målet med dagen er, og begyndte derefter arbejdet. 

Efter en brainstorm i starten af projektet, uddelte vi vores fokuspunkter mellem gruppemedlemmerne. Når man havde beskrevet et fokusområde, videregav man sin tekst til et andet gruppemedlem som så rettede den igennem. 

For at holde styr på projektet, startede vi hver gruppesession med at fortælle om hvad vi hver især havde arbejdet med. Dette sørgede for at vi havde et overblik over hvor langt vi var, samt vi alle fik forståelse for alle afsnit i rapporten.

Måden vi har arbejdet på i projektet, var ved at sige at vi var på universitetet fra 8.15-16.15 alle hverdage. Det vil sige at hvis der ikke var skemalagt forelæsninger en dag, var vi i grupperummet hele dagen, og hvis der var en forelæsningen om morgenen, var vi i grupperummet efter middag.

I gruppen var der en person der tog styringen, og sikrede sig at tingene blev lavet. Denne person sørgede for at alle filerne blev sat sammen i det færdige dokument, så vi kunne følge rapporten, og holde styr på hvor meget vi manglede at lave.

\subsection*{Gruppesamarbejde}
Vi startede som en gruppe på syv mennesker, det syvende medlem stoppede desværre med at dukke op tidligt i projektet, og efterlod ingen kontaktoplysninger, så vi kunne komme i kontakt med ham. Derfor arbejde vi videre som en gruppe på seks mennesker, hvor vi ikke turde overlade ansvar til dette medlem, af frygt for at han ikke ville dukke op.

Vores frygt viste sig at holde stik, og Hejan dukkede ikke op flere gange i projektet. Vi snakkede med vores semmesterkoordinater om problemet, og blev informeret om at vi ikke skulle skrive hans navn på rapporten.  

I gruppen udarbejdede vi en samarbejdsaftale som alle aktive gruppemedlemmer skrev under på. Denne samarbejdsaftale kan ses i bilag 1. Vi udarbejdede samarbejdsaftalen i fællesskab og diskuterede hvert punkt før det blev tilføjet til aftalen. Konsekvenserne for at bryde samarbejdsaftalen var at man skulle medbringe bagværk til gruppen. 

Generelt har gruppen været god til at diskutere i fællesskab, og alle personer har budt ind i vores diskussioner og tilføjet til projektet. For at have alle dokumenter i gruppen delt mellem personerne, har vi brugt git-systemet på hjemmesiden GitHub. Vi har derfor haft versionsstyring af alle dokumenter, og alle har kunne rette i alle dokumenter. 

For at lære hinanden bedre at kende, og gøre alle personer trygge ved at kommunikere i gruppen, holdte vi en grillaften sammen med vores tilknyttede tutor. 

\subsection*{Samarbejde med vejleder}
Til hvert møde med vejlederen sendte vi vores foreløbige arbejdsblade, samt hvilke spørgsmål vi gerne ville have besvaret til mødet. Før mødet med vejlederen holdte vi i gruppen et møde, hvor vi snakkede om spørgsmål der var dukket op, som vi også gerne ville have besvaret til mødet. 

Hen mod projektets ende, skete der en misforståelse mellem gruppen og vejlederen, hvilket førte til at der blev skrevet et afsnit der ikke kunne bruges i rapporten, dette blev dog hurtigt rettet op på, til næste møde.

\subsection*{Udarbejdelse af problemformulering}
For at udarbejde problemformuleringen, holdte vi på gruppen et møde, hvor vi diskuterede eventuelle problemformuleringer ud fra vores projektafgrænsning. Vi fandt i første omgang frem til en problemformulering, som vi sendte videre til vejlederen. Ud fra responsen på denne problemformulering ændrede vi denne til en mere generel problemstilling, som dog stadig var indskrænket i forhold til vores initierende problem.

\subsection*{Rapportstrukturering}
Rapporten er generelt struktureret efter den gængse rapportmodel: indledning \textrightarrow problemanalyse \textrightarrow projektafgrænsning \textrightarrow problemformulering. 

For at danne den røde tråd holdte vi et gruppemøde, hvor vi diskuterede hvordan de forskellige delafsnit skulle placeres, for at danne en rød tråd i rapporten.

Vi skrev rapporten i \LaTeX, hvilket der kun var \'en person som tidligere havde brugt. Dette skabte en smule forvirring i starten, men alle mennesker fik det lært.