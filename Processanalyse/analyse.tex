I gruppearbejdet var Hejan meget stille, mens vi andre tog diskussionerne op på gruppen. Som gruppe skulle vi måske have forsøgt at inddrage Hejan mere i diskussionerne, og vise ham at vi gerne ville have hans input med i gruppen, og at alle meninger leder mod et bedre projekt.

For at have en bedre struktur i gruppearbejdet, og for at fjerne den spildtid vi havde i dette projekt, kunne det være en forbedring at have en tidsplan. I starten af projektet snakkede vi om hvilke fokusområder vi kunne beskæftige os med inden for emnet, under udarbejdningen af vores rapport. Vi kunne i den forbindelse have taget skridtet videre, og sætte disse emner ind i en tidsplan med deadlines, for at få hele tiden at have overblik over hvad vi mangler at skrive. Hver person i gruppen kunne også have brugt tidsplanen til at finde ud af hvilke emner personen kunne skrive om, hvis man er blevet færdig med sit aktuelle emne.

En anden måde vi kunne have afhjælpet den manglende struktur i gruppen, var ved at have lavet et skema over vores dage i grupperummet. I dette projekt arbejde vi som sagt i grupperummene indtil klokken 16.15, men der var nogle gange nogle lidt for lange pauser i dette tidsrum. Vi havde ikke fastlagt hvornår pauserne skulle ligge, så omkring klokken 12 begyndte folk at holde pauser, men disse pauser blev ofte lange, da der ikke var nogen klar regel om hvornår arbejdet skulle genoptages. Her ville det have været fordelagtigt hvis vi i vores samarbejdsaftale havde skrevet hvornår disse pauser skulle ligge.

Misforståelsen der skete med vejlederen var at vi var kommet til at skrive et afsnit i rapporten der mindede mere om et teoriafsnit end en projektafgrænsning. Efter at vejlederen havde påpeget dette, kunne vi nå at rette det inden aflevering. Det virkede rigtig godt at vi havde lagt vores sidste møde med vejlederen 3 dage før aflevering, da dette gav os rig mulighed for at nå at ændre på de dele i vores rapport der havde brug for dette. Det tvang os også til at få skrevet meget af rapporten tidligt, hvilket gjorde at vi ikke fik travlt til sidst.

De aftaler der blev stillet i gruppen, var altid på baggrund af samtaler, og var derfor realistiske aftaler. Hvis aftalerne der blev foreslået var for omfattende, var der ikke noget problem ved at sige dette, og aftalen blev rettet til. Dette gjorde at de aftaler som blev besluttet, altid blev udført, da man selv var en stor del af beslutningsprocessen. Det havde en positiv effekt på gruppen at aftalerne blev udført, da at man så føler at alle mennesker i gruppen ønsker at projektet lykkes.

Da vi brugte både git og \LaTeX\ skulle en stor del af gruppen til at sætte sig ind i et nyt system, men da næsten alle var nye til disse systemer, løb vi også ofte ind i de samme problemer. Det gjorde at de fleste problemer hurtigt blev løst i gruppen, og at folk var hurtige til at kunne arbejde på denne nye måde.