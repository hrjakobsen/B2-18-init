\documentclass{article}
\usepackage{graphicx,hyperref,amsmath,natbib,bm,url}
\usepackage{microtype,todonotes}
\usepackage[a4paper,text={14.5cm,23.2cm},centering]{geometry}
\usepackage[compact,small]{titlesec}
\usepackage[utf8]{inputenc}
\usepackage[nottoc,numbib]{tocbibind}

\clubpenalty = 10000
\widowpenalty = 10000
\usepackage[T1]{fontenc}
\hypersetup{
     colorlinks   = true,
     citecolor    = black,
     linkcolor    = black,
     urlcolor     = black
}
\renewcommand{\figurename}{Figur}
\renewcommand{\contentsname}{Indholdsfortegnelse}

\begin{document}
	\title{Referat af møde 4/9-2015}
	\author{Gruppe B2-18}
	\maketitle
	Til mødet snakkede vi blandt andet om hvordan vi kan analysere sikkerheden ved et system i en selvkørende bil. Vi snakkede om følgende fokuspunkter kunne tages i problemanalysen:
	\begin{itemize}
		\item Kan de hackes når de snakker sammen?
		\item Skal de kunne snakke sammen når de skal kunne køre tæt i byen?
		\item Software i biler
		\item Bugs i software
		\item Sikker software (ingen bugs)
		\item Kommunikation med andre billister
	\end{itemize}
	Desuden kan der også tages udgangspunkt i det etiske spørgsmål om hvem der skal slås ihjel, hvis det er software der skal tage beslutningen.

	\section*{Omkring problemanalysen}
	Vi skal starte ud med et meget bredt problem, som vi starter med at præcisere. Dette kunne være ved at beskrive hvad en selvkørende bil er, samt hvem kunderne til disse er. Vi kigger nærmere på hvilke problemer der spiller ind på disse, og spidser til sidst ind på hvad man kunne fokusere på til en løsning til problemet.

	Problemformuleringen skal være \textbf{klar og præcis} omkring et problem som vi har \textbf{bevist eksisterer}.

	\noindent \textbf{Til næste gang skal vi:}
	\begin{itemize}
		\item Finde noget læsestof
		\item Begynde at skrive lidt
	\end{itemize}

	\noindent \textbf{Næste møde er d. 10/9-2015 klokken 15}
\end{document}
